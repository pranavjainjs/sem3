\documentclass{article}
\usepackage{amsmath} % amsmath package facilitates writing mathematical formula.
\usepackage{hyperref} % hyperref package allows to create internal and external references.
\hypersetup{colorlinks=true, linkcolor=blue}

% this is the introductory section of the document.
% it contains document title, today's date and name of author.
\title{Hello World!}
\date{November 15, 2022}
\author{Pranav Jain}

% document begins here.
\begin{document}

    % \maketitle command generates a title on a separate title page.
    \maketitle
    % to remove page numbering
    \thispagestyle{empty}
    \section{Getting Started}

    \noindent\textbf{Hello World!}
    Today, I am learning \LaTeX. \LaTeX\ is a great program for writing
    math. I can write in line math such as $ a^2+b^2=c^2 $. I can also give equations
    their own space:

    % Pythagoras theorem enclosed in equation environment
    \begin{equation}
        \gamma^2+\theta^2=\omega^2 \label{eq:Pythagoras}
    \end{equation}

    \noindent
    "Maxwell's equations" are named for James Clark Maxwell and are as follows:

    % power of amsmath package is utilised here.
    % to align the equations and names of equations.
    \begin{align}
        \vec{\nabla} \cdot \vec{E} \quad &= 
        \quad\frac{\rho}{\epsilon_0} &&
        \text{Gauss's Law} 
        \label{eq:Gauss} \\
        \vec{\nabla} \cdot \vec{B} \quad &= 
        \quad 0 &&
        \text{Gauss's Law for Magnetism} 
        \label{eq:GaussMag} \\
        \vec{\nabla} \times \vec{E} \quad &= 
        \quad-\frac{\partial{\vec{B}}}{\partial{t}} &&
        \text{Faraday's Law of Induction} 
        \label{eq:Faraday} \\ 
        \vec{\nabla} \times \vec{B} \quad &=
        \quad \mu_0\left( \epsilon_0\frac{\partial{\vec{E}}}{\partial{t}} + \vec{J}\right) &&
        \text{Ampere's Circuital Law} 
        \label{eq:Ampere}
    \end{align}

    % \quad adds horizontal spacing and enhances spacing.
    % \vec introduces the vector symbol over a character.
    % \cdot and \times add . and x symbols respectively.
    % \partial is used to print partial derivative symbols.
    % & , && help in alignment.

    % power of hyperref package is utilised here to create references to the respective equations.
    % links are referred using \ref{eq:LINK} tag
    Equations \ref{eq:Gauss}, \ref{eq:GaussMag}, \ref{eq:Faraday}, and \ref{eq:Ampere} are some of the mostimportant in Physics.

    % power of amsmath package is utilised here to create the matrix and vectors.
    \section{What about matrix equations?}
    \[   
        \begin{pmatrix}                                                 % gives curly brackets
            a_{1,1} & a_{1,2} & \cdots & a_{1,n} \\ 
            a_{2,1} & a_{2,2} & \cdots & a_{2,n} \\ 
            \vdots & \vdots & \ddots & \vdots \\ 
            a_{m,1} & a_{m,2} & \cdots & a_{m,n}
        \end{pmatrix}
        \begin{bmatrix}                                                 % gives square brackets
            v_1 \\
            v_2 \\
            \vdots \\
            v_n
        \end{bmatrix}
        =
        \begin{matrix}                                                  % no brackets
            w_1 \\
            w_2 \\
            \vdots \\
            w_n
        \end{matrix}
    \]
        
    % \vdots, \cdots, \ddots print the required ellipsis.
\end{document}