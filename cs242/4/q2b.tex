\documentclass{article}

%packages used in this document
\usepackage{amsmath}

\begin{document}

    % to remove page numbering
    \thispagestyle{empty}

    % \[ \] is used to create an environment which centers the text inside it.

    \[ 
        \iiint\limits_V f(x,y,z)\,dV = F 
    \]
    
    \[ 
        \frac{dx}{dy} = 
        x^{\prime} = 
        \lim_{h\to0} \frac {f\left(x+h\right)-f\left(x\right)} {h} 
    \]
    % \prime used to print ' character.
    
    \[
        \left|x\right| = 
        \begin{cases}
            -x, & \quad{\text{if } x < \text{0}} \\ 
            x, & \quad{\text{if } x \geq \text{0}}
        \end{cases}
    \]
    % cases with grouping braces splits the condition into two lines.
    
    \[ 
        F\left(x\right) = A_0 +
        \displaystyle\sum_{n\mathop{=}1}^N
        \left\lbrack 
            A_n\cos\left(\frac{2\pi nx}{P}\right) + 
            B_n\sin\left(\frac{2\pi nx}{P}\right)
        \right\rbrack 
    \]
    
    \[
        \displaystyle\sum_{n} \frac{1}{n^s} = \prod_{p} \frac{1}{1-\frac{1}{p^s}}
    \]
    % \frac used to print fractions.

    \[ 
        m\ddot{x} + c\dot{x} + kx = 
        F_0\sin\left(2\pi ft\right) 
    \]  
    % \dot and \\ddot print dot(s) over symbols.
    
    \[ 
        \begin{split}
            f(x)\quad   &= \quad x^2 + 3x + 5x^2 +8 +6x\\
                        &= \quad 6x^2 +9x +8\\
                        &= \quad x(6x+9)+8
        \end{split} 
    \]
    % & used to adjust alignment.
    % \quad used to adjust spacing.
    
    \[
        X = 
        \frac{F_0}{k} 
        \frac{1}{\sqrt{\left(1-r^2\right)^2 + \left(2\zeta r\right)^2}} 
    \]
    % \sqrt encapsulates the text under radical symbol.
    
    \[ G_{\mu\nu}\equiv R_{\mu\nu}-\frac{1}{2}Rg_{\mu\nu}=\frac{8\pi G}{c^4}T_{\mu\nu} \]
    
    \[ 
        6\text{CO}_2+6\text{H}_2\text{O} 
        \to 
        \text{ C}_6\text{H}_{12}\text{O}_6+6\text{O}_2
    \]
    % \text used to prevent the letters from being italicized.
    % \to prints arrow symbol.
    
    \[ 
        \text{SO}_4^{2-} + \text{Ba}^{2+}
        \to
        \text{BaSO}_4 
    \]
    
    \[   
        \begin{pmatrix}
            a_{1,1} & a_{1,2} & \cdots & a_{1,n} \\ 
            a_{2,1} & a_{2,2} & \cdots & a_{2,n} \\ 
             \vdots & \vdots & \ddots & \vdots \\ 
            a_{m,1} & a_{m,2} & \cdots & a_{m,n}
        \end{pmatrix}
        \begin{pmatrix}
            v_1 \\
            v_2 \\
            \vdots \\
            v_n
        \end{pmatrix}
        =
        \begin{pmatrix}
            w_1 \\
            w_2 \\
            \vdots \\
            w_n
        \end{pmatrix} 
    \]
    % \vdots, \cdots, \ddots print the required ellipsis.
    % \\ used to print the next row.
    % & used to adjust alignment.
    
    \[ 
        \frac{\partial \textbf{u}}{\partial t} + 
        \left(\textbf{u}\cdot\nabla\right)\textbf{u} - 
        \nu\nabla^2\left(\textbf{u}\right) = 
        -\nabla \textbf{h} 
    \] 
    % \textbf makes the closed text bold.
    
    \[ 
        \alpha A\beta B\gamma\Gamma\delta\Delta\pi\Pi\omega\Omega 
    \]
    
\end{document}